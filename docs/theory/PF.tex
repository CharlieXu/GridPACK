\documentclass[12pt]{article}
\usepackage{amssymb,amsmath}
\begin{document}
\title{Power Flow Equation Formulation}
\author{Yousu Chen \\PNNL}
\maketitle
This document is a description for how to formulate the power flow equation.
The current version is for Newton-Raphson AC power flow.

\subsection*{Basic Equations}
We start with the basic equations
\begin{eqnarray}
\label{eq:IYV}
\mathbf{I}&=&\mathbf{Y}\cdot\mathbf{V}\\
\label{eq:Ik}
I_{i}&=&\sum_{k=1}^{N}Y_{ik}V_{k}\\
\label{eq:Sk}
S{_i}&=& V{_i}I_{i}^{*} \nonumber \\
     &=& V{_i}\left ( \sum_{k=1}^{N}Y{_{ik}}V{_k} \right )^{*}  \nonumber \\          
     &=& V{_i}\sum_{k=1}^{N}Y{_{ik}}^{*}V{_k}^{*}
\end{eqnarray} 
where $N$ is the number of buses in the network, $\mathbf{V}$ is a vector of
voltages, $\mathbf{I}$ is a vector of currents and $\mathbf{Y}$ is the Y matrix.

Suppose we define the complex voltages as 
\begin{equation}
\label{eq:CXV}
V{_i}= |V{_i}|e^{j\theta_{i}}
\end{equation} 
and then define the phase angle between bus $i$ and bus $k$ as
\begin{equation}
\theta_{ik}=\theta_{i}-\theta_{k} \\
\end{equation} 
Further, we can decompose the elements of the Y matrix into real and imaginary
parts using
\begin{equation}
Y_{ik}=G_{ik}+jB_{ik}
\end{equation} 
The $G_{ik}$ are called conductances and the $B_{ik}$ are called susceptances. 

Using these definitions we can write the expressions for the quantity
$\mathbf{S}$ as
\begin{equation}
\label{eq:S}
S_{i}=\sum_{k=1}^{N}|V{_i}||V{_k}|e^{j\theta_{ik}}\left ( G_{ik}-jB_{ik} \right )
\end{equation} 
We can further decompose the $S_i$ into real and imaginary parts using
\begin{equation}
S_{i}=P_{i}+jQ_{i}
\end{equation} 
This can be rewritten as
\begin{equation}\label{eq:PQ}
P{_i}+jQ{_i}=\sum_{k=1}^{N}|V{_i}||V{_k}|\left ( \cos\theta_{ik}+j\sin\theta_{ik} \right )\left ( G_{ik}-jB_{ik} \right )
\end{equation} 
Resolving equation (\ref{eq:PQ}) into real and imaginary parts gives the
expressions
\begin{equation}\label{eq:PiQi}
\begin{cases}
P{_i}=\sum_{k=1}^{N}|V{_i}||V{_k}|\left (G_{ik}\cos\theta_{ik}+B_{ik}\sin\theta_{ik} \right ) \\
Q{_i}=\sum_{k=1}^{N}|V{_i}||V{_k}|\left (G_{ik}\sin\theta_{ik}-B_{ik}\cos\theta_{ik} \right )
\end{cases}
\end{equation} 

\subsection*{Newton-Raphson AC Power Flow}

Now set up power flow equations in the form of $\mathbf{f}\left (\mathbf{x} \right )=\mathbf{0}$.
The voltage magnitude and phase angle of the slack/reference bus ($\theta_{1}$
and $|V_{1}|$) are known.
\newline
\newline
\noindent
Define
\begin{eqnarray}
\theta &=&\left [ \begin{array}{c} \theta{_2}\\.\\.\\.\\ \theta_{N} \end{array} \right] \\
|\mathbf{V}| &=&\left [ \begin{array}{c} |V_{2}| \\.\\.\\.\\|V_{N}| \end{array} \right]\\
\mathbf{x}  &=&\left [ \begin{array}{c} \boldsymbol{\theta}\\|\mathbf{V}| \end{array} \right]
\end{eqnarray}
The form of the vector function $\mathbf{f}(\mathbf{x})$ is
\begin{equation}
\label{eq:RHS}
\mathbf{f}\left (\mathbf{x} \right ) = \left [ \begin{array}{c} P_{2}\left (
\mathbf{x} \right ) - P_{2} \\.\\.\\.\\ P_{N}\left ( \mathbf{x} \right ) - P_{N}
\\-------\\Q_{2}\left ( \mathbf{x} \right ) - Q_{2} \\.\\.\\.\\ Q_{N}\left (
\mathbf{x} \right ) - Q_{N} \end{array} \right ] =- \left [ \begin{array}{c} \Delta \mathbf{P}\left ( \mathbf{x} \right ) \\ \Delta \mathbf{Q}\left ( \mathbf{x} \right ) \end{array} \right ] = \mathbf{0}
\end{equation}
where 
\begin{equation}\label{eq:DeltaPQ}
\begin{cases}
\Delta\mathbf{P} = P_{i} - P_{i}\left ( \mathbf{x} \right )  \\
\Delta\mathbf{Q} = Q_{i} - Q_{i}\left ( \mathbf{x} \right ) 
\end{cases}
\end{equation}
$P_{i}$ and $Q_i$ can be obtained from the equation (\ref{eq:PiQi}).

Now consider the Jacobian of $\mathbf{f}$, $\mathbf{J}$. The dimension of
$\mathbf{J}$ using the expression for the complex voltage (\ref{eq:CXV}) is
$(2(N-1))(2(N-1))$. The Jacobian itself has the form
\begin{eqnarray}
\mathbf{J}  = \left [ \begin{array}{cc} \mathbf{J1} & \mathbf{J2} \\ \mathbf{J3} & \mathbf{J4} \end{array} \right ] \nonumber \\
\label{eq:JBN}
            = \left [ \begin{array}{cc} \frac{\partial \mathbf{P}}{\partial \boldsymbol{\theta}} & \frac{\partial \mathbf{P}}{\partial \mathbf{|V|}} \\ \frac{\partial \mathbf{Q}}{\partial \boldsymbol{\theta}} & \frac{\partial \mathbf{Q}}{\partial \mathbf{|V|}}  \end{array} \right ]
\end{eqnarray}
The N-R iteration itself has the form
\begin{equation}\label{eq:PFE}
\mathbf{J}\mathbf{\Delta x} = - \mathbf{f} \left ( \mathbf{x} \right )
\end{equation}
Combining equations (\ref{eq:RHS}), (\ref{eq:JBN}) and (\ref{eq:PFE}) gives the
expression
\begin{equation}\label{eq:PFE2}
\left [ \begin{array}{cc} \mathbf{J1} & \mathbf{J2} \\ \mathbf{J3} & \mathbf{J4} \end{array} \right ] \left [ \begin{array}{c} \mathbf{\Delta \theta} \\ \mathbf{\Delta |V|} \end{array} \right ] =  \left [ \begin{array}{c} \mathbf{\Delta P} \left (\mathbf{x} \right ) \\  \mathbf{\Delta Q} \left (\mathbf{x} \right ) \end{array} \right ]
\end{equation}

For the $n^{th}$ iteration, we can rearrange equation (\ref{eq:PFE2}) to get
\begin{equation}\label{eq:update}
\left [ \begin{array}{c} \theta^{n+1} \\ |V|^{n+1} \end{array} \right ] = \left [ \begin{array}{c} \theta^{n} \\ |V|^{n} \end{array} \right ] -  \left [ \mathbf{J} \left ( x^{n} \right ) \right] ^{-1}  \left [ \begin{array}{c} \Delta P\left ( x^{n} \right ) \\  \Delta Q\left ( x^{n} \right )  \end{array} \right ] 
\end{equation}

The expressions for the elements of $\mathbf{J}$ can be found below:
\begin{equation}\label{eq:J1} 
J1 =\left\{\begin{matrix}

\frac{\partial P_i}{\partial \theta_i} = -\sum_{k\ne i}{|V_i||V_k|\left (G_{ik}\sin\theta_{ik} - B_{ik}\cos\theta_{ik}\right )} &\\
\frac{\partial P_i}{\partial \theta_k} = |V_i||V_k|\left (G_{ik}\sin\theta_{ik} - B_{ik}\cos\theta_{ik}\right )    & i\ne k

\end{matrix}\right .
\end{equation} 


\begin{equation}\label{eq:J2}
J2 =\left\{\begin{matrix}

\frac{\partial P_i}{\partial |V_i|} = 2|V_i|G_{ii} + \sum_{k\ne i}{|V_k|  \left (G_{ik}\cos\theta_{ik} + B_{ik}\sin\theta_{ik}\right )} &\\
\frac{\partial P_i}{\partial |V_k|} = |V_i|\left ( G_{ik}\cos\theta_{ik} +  B_{ik}\sin\theta_{ik}\right )    & i\ne k

\end{matrix}\right  .
\end{equation} 


\begin{equation}\label{eq:J3}
J3 =\left\{\begin{matrix}

\frac{\partial Q_i}{\partial \theta_i} = \sum_{k\ne i}{|V_i||V_k|\left (G_{ik}\cos\theta_{ik} + B_{ik}\sin\theta_{ik}\right )} &\\
\frac{\partial Q_i}{\partial \theta_k} = -|V_i||V_k|\left (G_{ik}\cos\theta_{ik} + B_{ik}\sin\theta_{ik}\right )    & i\ne k

\end{matrix}\right .
\end{equation} 

\begin{equation}\label{eq:J4}
J4 =\left\{\begin{matrix}

\frac{\partial Q_i}{\partial |V_i|} = -2|V_i|B_{ii} + \sum_{k\ne i}{|V_k|  \left (G_{ik}\sin\theta_{ik} - B_{ik}\cos\theta_{ik}\right )} &\\
\frac{\partial Q_i}{\partial |V_k|} = |V_i|\left ( G_{ik}\sin\theta_{ik} -  B_{ik}\cos\theta_{ik}\right )    & i\ne k

\end{matrix}\right .
\end{equation} 

The diagonal elements of $J1$ - $J4$ can also be written in the following form:
\begin{equation}\label{eq:J1D}
J1_{diag} = \frac{\partial P_i}{\partial \theta_i} = -Q_i - B_{ii}|V_i|^2
\end{equation} 
\begin{equation}\label{eq:J2D}
J2_{diag} = \frac{\partial P_i}{\partial |V_i|} = P_i - G_{ii}|V_i|^2
\end{equation} 
\begin{equation}\label{eq:J3D}
J3_{diag} = \frac{\partial Q_i}{\partial \theta_i} = \frac{P_i}{|V_i|} + G_{ii}|V_i|
\end{equation} 
\begin{equation}\label{eq:J4D}
J4_{diag} = \frac{\partial Q_i}{\partial |V_i|} = \frac{Q_i}{|V_i|} - B_{ii}|V_i|
\end{equation} 

where $P_i$ and $Q_i$ are given by equation (\ref{eq:PiQi}).

The solution strategy is:
\begin{enumerate}
\item Assign initial voltage magnitudes and zero phase angles $\left ( \theta \right )$
to all buses. For PQ buses(load buses), the voltage magnitudes are set equal to 1.0;
for PV buses (voltage-controlled/generator buses), the voltage magnitudes are specified.
\item Calculate $P_{i}$, $Q_i$ using equation (\ref{eq:PiQi}).
\item Calculate $\Delta P_{i}$, $\Delta Q_{i}$ using equation (\ref{eq:DeltaPQ}).
\item Calculate the element of Jacobian $\mathbf{J}$ using equation (\ref{eq:J1}) to (\ref{eq:J4}).
\item Solve the linear equation of (\ref{eq:PFE2}) to find $\Delta \theta$ and $\Delta |V|$.
\item Update new voltage magnitudes and phase angles using equation (\ref{eq:update}).
\item Continue until the residuals of $\Delta P$ and $\Delta Q$ are less than the specified tolerance. 
\end{enumerate}
\end{document}

