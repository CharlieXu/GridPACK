\documentclass[12pt]{article}
\begin{document}
\title{YBUS Admittance Matrix Formulation}
\author{Yousu Chen \\PNNL}
\maketitle
This document is a description of how to formulate the YBUS admittance matrix.
In general, the diagonal terms $Y_{ii}$ are the self admittance terms and are equal to
the sum of the admittances of all devices incident to bus $i$.  The off-diagonal
terms $Y_{ij}$ are equal to the negative of the sum of the admittances joining
the two buses. Shunt terms only affect the diagonal entries of the $Y$ matrix.
For large systems, $Y$ is a sparse matrix and it is structually symmetric. 

\subsection*{Transmission Lines}
Transmission lines run from bus $i$ to bus $j$ and are indexed by $k$. More than
one transmission line can go between two buses.
\begin{eqnarray}
Y_{ij} &=& \sum_{k}\frac{-1}{r_{ijk}+j\*x_{ijk}} \\
Y_{ji} &=& Y_{ij} \\
Y_{ii} &=& - \sum_{j\ne i} Y_{ij} \\
Y_{jj} &=& - \sum_{i\ne j} Y_{ji}
\end{eqnarray} 
where:\\*
$Y_{ij}$: the ${ij}_{th}$ element in the Y matrix.\\* 
$i$: the ``from'' bus.\\*
$j$: the ``to'' bus. \\*
$k$: the $k_{th}$ transmission line from $i$ to $j$. \\*
$r_{ijk}$: the resistance of the $k_{th}$ transmission line from $i$ to $j$.\\*
$x_{ijk}$: the reactance of the $k_{th}$ transmission line from $i$ to $j$.

\subsection*{Transformers}
For a tap changing and phase shifting transformer, the off-nominal tap value can
in general be considered as a complex number $a$, where the tap ratio is $t$
and the phase shift is $\theta$. Transformers are defined similarly to
transmission lines and exist on a branch between bus $i$ and bus $j$.
Therefore, all transformer elements are subscripted by $i$ and $j$. For
simplicity, we assume that only one transformer line exists between two buses
and if a transformer is present between two buses then no transmission lines
exist between the two buses. The off-nominal tap ratio between buses $i$ and $j$
is defined as $a_{ij} = t_{ij}*(\cos{\theta_{ij}}+j*\sin{\theta_{ij}})$.
We can define
\begin{eqnarray}
y_{ij}^{t} &=& \frac{-1}{r_{ij}+j\*x_{ij}} \\
Y_{ij} &=& -\frac{y_{ij}^{t}}{a_{ij}^{*}} \\
Y_{ji} &=& -\frac{y_{ij}^{t}}{a_{ij}} \\
Y_{ii} &=& \frac{y_{ij}^{t}}{{|a_{ij}|}^2} \\
Y_{jj} &=& y_{ij}^{t}
\end{eqnarray} 
where:\\*
$Y_{ij}$: the ${ij}_{th}$ element in the $Y$ matrix.\\* 
$i$: the ``from'' bus.\\*
$j$: the ``to'' bus. \\*
$r_{ij}$: the resistance of the transformer between $i$ and $j$.\\*
$x_{ij}$: the reactance of the transformer between $i$ and $j$.\\*
$t_{ij}$: the tap ratio between bus $i$ and bus $j$.\\*
$\theta_{i}$: the phase on bus $i$.\\*
$\theta_{j}$: the phase on bus $j$.\\*
$\theta_{ij}=\theta_{i}-\theta_{j}$: the phase shift from bus $i$ to bus $j$.\\*
$a_{ij}^*$: the conjugate of $a_{ij}$.\\*
\\*
Given the bus admittance matrix $Y$ for the entire system, the transformer model
can be introduced by modifying the elements of the Y-matrix derived from the
transmission lines as follows:
\begin{eqnarray}
Y_{ij}^{new} &=&  - \frac{y_{ij}^{t}}{a_{ij}^{*}} \\
Y_{ji}^{new} &=&  - \frac{y_{ij}^{t}}{a_{ij}} \\
Y_{ii}^{new} &=& Y_{ii} + \frac{y_{ij}^{t}}{{|a_{ij}|}^2} \\
Y_{jj}^{new} &=& Y_{jj} + y_{ij}^{t}
\end{eqnarray} 
Note that since we are assuming that a branch with a transformer on it does not
carry any additional transmission lines, the $Y_{ij}^{new}$ do not have any
contributions from $Y_{ij}$.
\subsection*{For shunts}
Shunts only contribute to diagonal elements. The sources of shunts include:
\begin{itemize}
\item shunt devices located at buses $(g_{i}^{s}+j\*b_{i}^{s})$;
\item transmission line/transformer charging $b_{ijk}$ (distributed half to each end)
     from end: $b_{ik}$= $0.5\*b_{ijk}$;
     to end: $b_{jk}$= $0.5\*b_{ijk}$;
\item transmission line/transformer shunt admittance, which is normally a small
value: $(g_{ijk}^{a}+j\*b_{ijk}^{a})$; the shunt admittance contributes the same
amount to both ends of the line
\end{itemize}
Therefore, the general equation for diagonal elements is:
\begin{equation}
Y_{ii}^{tot} = - (\sum_{j\ne i} Y_{ij}^{new})
+ g_{i}^{s}+j\*b_{i}^{s}
 + \sum_k (j\*b_{ki} + g_{ki}^{a}+j\*b_{ki}^{a})
\end{equation} 
\begin{equation}
Y_{jj}^{tot} = - (\sum_{i\ne j} Y_{ji}^{new})
 + g_{j}^{s}+j\*b_{j}^{s}
 + \sum_k (j\*b_{kj} + g_{kj}^{a}+j\*b_{kj}^{a})
\end{equation} 
where:\\*
$Y_{ij}^{tot}$: the ${ij}_{th}$ element in the Y matrix.\\* 
$i$: the ``from'' bus. \\*
$j$: the ``to'' bus. \\*
$k$: the $k_{th}$ transmission line/transformer from $i$ to $j$. \\*
$g_{i}^{s}+j*b_{i}^{s}$: the shunt at bus $i$. \\*
$b_{ijk}$: the line charging of the $k_{th}$ line. \\*
$b_{ik}$ = 0.5 *$b_{ijk}$ : the line charging of the $k_{th}$ line assigned to "from" end $i$. \\*
$b_{jk}$ = 0.5 *$b_{ijk}$ : the line charging of the $k_{th}$ line assigned to "to" end $j$. \\*
$g_{ki}^{a}+b_{ki}^{a} = g_{ijk}^{a}+j\*b_{ijk}^{a}$: the $k_{th}$ line shunt admittance at "from" end $i$. \\*
$g_{kj}^{a}+b_{kj}^{a} = g_{ijk}^{a}+j\*b_{ijk}^{a}$: the $k_{th}$ line shunt admittance at "to" end $j$. \\*


\begin{center}
\textit{Thanks!}
\end{center}
\end{document}

