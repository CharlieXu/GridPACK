\documentclass[12pt]{article}
\renewcommand{\baselinestretch}{1.0}
\begin{document}
\begin{titlepage}
\begin{center}
{\LARGE DRAFT: Interface Specification for PGRID Module}
\end{center}
\end{titlepage}
%\tableofcontents
\newpage
\pagestyle{plain}
\section{Introduction}
This module is designed to create a network, partition it across multiple
processors, and allow applications to add and modify fields to the different
buses and branches.
\section{Initializing and terminating the PGRID module}
A small class PGRIDFactory is used to initialize and terminate the PGRID module.
\subsection{Init}
\begin{verbatim}
void PGRIDFactory::Init(void)
\end{verbatim}
This function initializes the PGRID module. It must be called before any other
functions in the PGRID module can be called and must be called on all
processors.
\subsection{Terminate}
\begin{verbatim}
void PGRIDFactory::Terminate(void)
\end{verbatim}
This function closes down the PGRID module and cleans up all utility data
structures. No PGRID module functions can be called after this routine is
called. This function must be called on all processors.

\section{Creating individual power grid networks}
Power grids are represented as instances of the class PGRID. The following
functions in this section can be used to create and instance of a power grid,
add branches and buses to the network and partition the power grid across
processors.
\subsection{PGRID}
\begin{verbatim}
PGRID::PGRID(void)
\end{verbatim}
Constructor for the PGRID class. This can be used to create an individual
instance of a power grid.
\subsection{$\sim$PGRID}
\begin{verbatim}
PGRID::~PGRID(void)
\end{verbatim}
Destructor for the PGRID class. This cleans up all fields associated with a
power grid instance and removes the power grid itself.
\subsection{SetGroup}
\begin{verbatim}
void PGRID::SetGroup(int nproc,
                      int *list)
\end{verbatim}
\begin{itemize}
\item (IN) \texttt{nproc}: number of processors in group
\item (IN) \texttt{list}: list of processors in group
\end{itemize}
This function can be used to restrict a power grid instance to a subset of
processors. This function must be called on all processors in the processor
group.

\subsection{AddBranch}
\begin{verbatim}
void PGRID::AddBranch(int ibus1,
                      int ibus2)
\end{verbatim}
\begin{itemize}
\item (IN) \texttt{ibus1}: global index of bus 1 at end of branch
\item (IN) \texttt{ibus2}: global index of bus 2 at end of branch
\end{itemize}
This function adds a branch to the network.  The branch is defined by the global
indices \texttt{ibus1} and \texttt{ibus2} that define the two buses at the
end of the branch.
\subsection{AddBus}
\begin{verbatim}
void PGRID::AddBus(int bus_id)
\end{verbatim}
\begin{itemize}
\item (IN) \texttt{bus\_id}: global index of bus
\end{itemize}
This function adds a bus to the network. The global index \texttt{bus\_id}
must represent a unique integer index for the bus but does not need to be part
of a sequence of consecutive numbers. The partioning function will derive a
unique integer index that represents an internal ordering of the buses.
\subsection{SetReferenceBus}
\begin{verbatim}
void PGRID::SetReferenceBus(int bus_id)
\end{verbatim}
\begin{itemize}
\item (IN) \texttt{bus\_id}: global index of bus
\end{itemize}
This function sets the reference bus. It is called on whatever processor holds
the reference bus, before the buses are partitioned across processors.
\subsection{Partition}
\begin{verbatim}
void PGRID::Partition(void)
\end{verbatim}
This function partitions the grid so that it is distributed evenly across all
processors. After calling this, it is no longer possible to add buses and
branches to the grid or set the reference bus.

\section{Accessing local grid information}
After the processor grid is partitioned, each processor will contain a
subsection of the grid. This may be either completely connected or it may be
composed of several disconnected pieces. The functions in this section describe
routines for accessing the number of buses and branches representing the portion
of the network held locally on the processor.
\subsection{GetProp}
\begin{verbatim}
int PGRID::GetProp(char *prop_name)
\end{verbatim}
\begin{itemize}
\item (IN) \texttt{prop\_name}: name of the requested property
\end{itemize}
This function return the value of a grid property specified by the string. For
example, the requested property could be \texttt{"num\_buses"}, in which case
the function returns the number of buses associated with the processor.
Similarly, a different string would return the number branches associated with the
processor.
\subsection{GetBranchBuses}
\begin{verbatim}
int PGRID::GetBranchBuses(int local_branch_index,
                          int *bus1,
                          int *bus2)
\end{verbatim}
\begin{itemize}
\item (IN) \texttt{local\_branch\_index}: local index of the branch
\item (OUT) \texttt{bus1}: local index of the bus at one end of the branch
\item (OUT) \texttt{bus2}: local index of the bus at the other end of the
branch
\end{itemize}
This function will return the local bus indices of the two buses at either end
of a branch. The function returns 0 if the value of \texttt{local\_branch\_index} is
outside the range of values found on the calling processor.
\subsection{GetBusBranches}
\begin{verbatim}
int PGRID::GetBusBranches(int local_bus_index,
                          int *num_branches,
                          int *branch_list)
\end{verbatim}
\begin{itemize}
\item (IN) \texttt{local\_bus\_index}: local index of the bus
\item (OUT) \texttt{num\_branches}: the number of branches attached to this bus
\item (OUT) \texttt{branch\_list}: a list of the local branch indices for the
\texttt{num\_branches} branches attached to this bus.
\end{itemize}
This function returns the number of branches attached to a bus and a list of the
local bus indices for the attached branches. The function returns 0 if the value
of \texttt{local\_bus\_index} is outside the range of values found on the
calling processor.

\section{Adding (and deleting?) bus and branch fields. Accessing branch field
elements}
Probably need some functions that can add something besides scalar fields to
buses and branches.
\subsection{AddField}
\begin{verbatim}
void PGRID::AddField(char *field_type,
                     char *data_type,
                     char *field_name,
                     PGRIDField **field)
\end{verbatim}
\begin{itemize}
\item (IN) \texttt{field\_type}: a character string specifying whether the field
is located on buses or branches
\item (IN) \texttt{data\_type}: a character string specifying data type of the
field elements
\item (IN) \texttt{field\_name}: a character string providing a unique
identifier for the field
\item (OUT) \texttt{field}: a pointer to the newly created field object
\end{itemize}
This function can be used to create a new field on the grid network. The
\texttt{field\_type} variable can be used to distinguish between bus and branch
fields and the \texttt{data\_type} variable can be used to specify if the field
represents, single or double precision real numbers, single or double precision
complex numbers, or integers. (DO WE NEED TO INCLUDE BOOLEAN FIELDS?) The field
name is a character string that can be
assigned to the field and used to retrieve it from the \texttt{PGRID} module in
later operations. The function returns a \texttt{PGRIDField} field object that
can be used to set or modify individual values of the field.

\section{Accessing field properties}
Accessing or modifing properties of individual elements
\begin{itemize}
\item \begin{verbatim}
void PGRID::GetElem(int elem_index,
                    PGRIDField *field,
                    void *value)
\end{verbatim}
\item \begin{verbatim}
void PGRID::SetElem(int elem_index,
                    PGRIDField *field,
                    void *value)
\end{verbatim}
\item \begin{verbatim}
void PGRID::UpdateGrid(void)
\end{verbatim}
\item \begin{verbatim}
void PGRID::UpdateGridField(PGRIDField *field)
\end{verbatim}
\end{itemize}
\end{document}
